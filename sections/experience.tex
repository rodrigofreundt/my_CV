
\section{Experience}

\outerlist{

\entrybig
	{\textbf{CCAT collaboration}}{Ithaca, NY}
	{\textit{Graduate Research Assistant.} Advisors: Gordon Stacey and Michael Niemack}{2021 \textendash Present}
\innerlist{
	\entry{Mechanical design and fabrication of the Epoch of Reionization Spectrometer (EoR-Spec). EoR-Spec has been designed to probe the redshifted [C II] 158 $\mu$m fine-structure line emission from aggregates of galaxies in the early universe (z = 3.5 to 8).}
	\entry{Characterization of detector array of more than 6000 Kinetic Inductance Detectors (KIDs) operating at 100 mK.}
	\entry{Fabry-Perot Interferometer (FPI) that covers the full spectral range of 210-420 GHz with a resolving power of R $\sim$ 100.}
	\entry{Development of a high density cryogenic readout harness for the Prime-Cam instrument, capable of reading out more than 20,000 KIDs.}
}

\entrybig
	{\textbf{Atacama Cosmology Telescope}}{San Pedro de Atacama, Chile}
	{Site Engineer. Supervisors: Prof. Suzanne Staggs and Prof. Mark Devlin.}{Oct 2019 \textendash Jul 2021}
\innerlist{
	\entry{Ensured the continuous and smooth operation of all of the telescope's subsystems, such as a the cryogenic cooling system (\raisebox{-0.75ex}{\textasciitilde}90mK). }
	\entry{Performed calibration procedures like the alignment of the receiver and the primary and secondary reflectors through high-precision photogrammetry (\raisebox{-0.75ex}{\textasciitilde}10\(\mu\)m). Carried out frequency response measurements of the detector arrays through Fourier Transform Spectroscopy.}
	\entry{Participated in the deployment of the Advance ACTPol low-frequency detector array (27/39 GHz). Assisted in the installation of the new chain of filters and cold silicon reimaging optics.}
  }

\entrybig
	{\textbf{Management Solutions}}{Lima, Peru}
	{Assistant Business Consultant}{Sep 2018 \textendash Jul 2019}
\innerlist{
	\entry{Worked in big data and bank stress test for a major local bank. Also worked in software development and data science for a leading financial institution in Madrid, Spain.}

}

\entrybig
	{\textbf{Institute for Radioastronomy}}{Lima, Peru}
	{Research Assistant. Supervisor: Prof. Jorge Heraud}{May 2017 \textendash Aug 2018}
\innerlist{
	\entry{Worked in the design and construction of an 8-meter diameter radio telescope (RT8). Designed several hardware and software systems for the telescope's electromechanical pointing system.}
 	\entry{Assisted in the design and characterization of an hydrogen 21-cm line feedhorn and its RF over Fiber (RFoF) frontend. Developed an antenna test system based on a custom-made positioning goniometer and software-defined radio technology from Ettus Research.}

}

}